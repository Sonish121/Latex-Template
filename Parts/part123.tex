
\section{\MakeUppercase{Introduction}}

    \subsection{Background}
    Traditionally, in the field of software development, the creation of user 
    interfaces (\gls{ui}) has relied heavily on manual processes. Developers 
    have been tasked with interpreting design specifications and translating 
    them into functional code. This method often involves manually coding \gls{ui} elements such as buttons, input fields, menus, and layouts. While this approach has been effective in bringing designs to life, it is both time-consuming and prone to introducing errors. The discrepancies between the designer’s vision and the developer’s implementation can result in interfaces that deviate significantly from the original design. Recognizing these limitations, the "Sketch-to-Code" system proposes a new approach that automates the conversion of design sketches into executable front-end code, thereby addressing these persistent challenges.
    The reliance on manual coding for \gls{ui} development has long been a bottleneck in the software development process. Designers often go to great lengths to craft meticulously detailed and visually consistent design systems. However, these designs often fail to seamlessly translate into the development phase. The variations introduced during manual implementation not only waste valuable time but also lead to inconsistencies that can compromise the overall user experience. The "Sketch-to-Code" system aims to bridge this gap by providing a tool that leverages advanced technologies to automate the transformation of design sketches into functional code. This solution not only enhances efficiency but also ensures that the implemented interface remains faithful to the original design.
    
    Recent advancements in technology, particularly in the fields of \gls{ai} and machine learning, have significantly impacted the way \gls{gui} sketches are converted into functional code. Researchers and developers are increasingly exploring the potential of these technologies to streamline \gls{ui} development. By utilizing sophisticated algorithms and image processing techniques, automated systems are now capable of analyzing design sketches, interpreting their components, and generating corresponding code. These systems offer several key advantages. First, they drastically reduce the time required for \gls{ui} development, enabling developers to focus on more complex and creative tasks. Second, they ensure a higher level of consistency between the design and the implemented interface, thereby reducing the need for extensive revisions. Finally, they allow designers and developers to collaborate more effectively, as the automated system serves as a bridge that translates design specifications into code with minimal manual intervention.One of the most notable benefits of the "Sketch-to-Code" system is its ability to enhance the overall efficiency of the software development process. By automating repetitive and time-consuming tasks, developers can allocate their efforts toward refining the functionality and user experience of the application. This shift in focus not only accelerates project timelines but also results in higher-quality software products. Furthermore, the consistency ensured by the system eliminates many of the errors that typically arise during manual implementation, thereby reducing the need for extensive testing and debugging.The integration of \gls{ai} into \gls{ui} development marks a significant milestone in the evolution of software engineering practices. As these \gls{ai}-driven solutions continue to advance, they hold the promise of revolutionizing the way software is designed and developed. By making \gls{ui} development more accessible and efficient, these technologies have the potential to empower developers across a wide range of industries. Whether it is a small startup looking to prototype a new idea or a large enterprise aiming to streamline its development pipeline, the "Sketch-to-Code" system offers a powerful tool to achieve these goals.In conclusion, the "Sketch-to-Code" system represents a transformative approach to \gls{ui} development. By automating the conversion of design sketches into functional code, it addresses many of the challenges that have long plagued designers and developers. With the support of \gls{ai} and machine learning, this system not only enhances efficiency and consistency but also paves the way for more innovative and user-centric software solutions. As the technology continues to evolve, it is poised to become an integral part of modern software development, enabling teams to bring their creative visions to life with greater speed and precision.
    
    
    \subsection{Motivation}
    This project is focused on addressing a common and time-consuming issue in the software development process: transforming design ideas into fully functional software. Often, developers encounter significant challenges when the final product fails to align with the original design vision. These mismatches typically result in repeated adjustments and rework, which not only disrupts the development timeline but also leads to inefficiencies and delays in project completion. To tackle this issue, our approach involves providing a simple and effective solution in the form of prototype front-end code. By automating the initial steps of creating user \gls{ui}, we aim to streamline this process, making it quicker and more efficient for developers to get started on their projects.By leveraging automation, this project seeks to reduce the technical burden associated with manually building \gls{ui} components such as buttons, menus, and layouts. Instead of spending valuable time on these repetitive and detailed tasks, developers will have the opportunity to focus on enhancing the overall functionality and user experience of the software. This shift not only accelerates the development process but also ensures that the final product remains true to the original design vision, minimizing discrepancies and the need for extensive revisions.

    The ultimate goal of this project is  to simplify and enhance the software development workflow. By providing developers with tools that bridge the gap between design and implementation, we aim to create a more seamless and productive development experience. This approach not only saves time but also ensures that the end product aligns closely with the intended design, eliminating unnecessary variations and preserving the integrity of the original vision. Through these advancements, this project aspires to empower developers to deliver high-quality software more efficiently and effectively.
    \subsection{Problem Definition}
The problem is to develop a system that can automate the process of turning sketches
into functional front-end code. The system should be able to craft interactive \gls{ui} that is more like designer intended.
Some of the key challenges are listed below:

\begin{itemize}
    \item \textbf{Ambiguity in Requirements:} Sketches may not always clearly define all
    requirements, leading to ambiguity. This can result in misunderstandings
    between stakeholders, designers, and developers.
    \item \textbf{Technical Feasibility:} Sometimes, what looks good on paper (or screen) may
    not be technically feasible within the constraints of the project, platform, or
    technology stack. 
    \item \textbf{Integration Complexity:} Projects often require integrating various
    components, APIs, or third-party services, which can introduce complexity
    beyond the initial sketch.
    \item \textbf{\gls{ux} Considerations:} Implementing a sketch into code
    involves not just visual fidelity but also ensuring a smooth and intuitive user
    experience, which may require iterative improvements.
\end{itemize}
    \subsection{Objectives}
    The main objectives of our project are listed below :
    \begin{itemize}
        \item To construct a model able to generate quick \gls{gui} prototype
        \item To make intutive user interface for customizing and stylizing the generated
code.
    \end{itemize}
    
    \subsection{Project Scope and Applications}
    \sloppy
    In the world of computer science, most projects start with a concept or idea. These ideas are often visualized through sketches, diagrams, or mockups. These sketches serve as an important part of the project’s foundation, helping to define the project's main goals, user interfaces, and the architecture of the system. They represent a high-level overview of how the final product should look and function. The main challenge is to take these static sketches and transform them into working, functional code that fulfills the intended goals of the project, aligns with the requirements, and meets user expectations.
    
    The scope of converting these sketches into code is a complex process. It involves more than just turning images into text-based code. The task requires ensuring that the design is not only visually accurate but also functionally correct. Developers need to ensure that the features and components shown in the sketch are possible to build with the given technology stack, and that they meet the performance and usability expectations. This process covers various areas, including the visual design, user interactions, back-end architecture, and integration of multiple systems or services. The ultimate goal is to create software that functions just as the designer intended, while also being practical, efficient, and responsive.The application of turning sketches into code has several key benefits for the development process. First, it provides a clear, structured path for developers to follow, making sure they know exactly what the final product should look like and how it should work. It allows them to better understand both the functional and aesthetic requirements of the project, which are critical for creating a seamless user experience. By following the sketches, developers can create software that is both visually appealing and user-friendly.When developers adhere to the sketches during the development process, they can ensure that the design remains consistent. Consistency is important because it maintains the intended look and feel of the application across all parts of the project. This consistency reduces the chances of errors or discrepancies arising between different sections of the software, ensuring that all elements work together harmoniously. By starting with clear sketches, the team can avoid confusion and misalignment during the development process, making it easier to make design decisions and solve problems quickly.
    
    In addition to providing direction for developers, sketches also act as a useful communication tool. They provide a visual representation that helps different stakeholders, such as designers, developers, and project managers, understand the project’s objectives. This shared understanding is crucial for minimizing misunderstandings and ensuring that everyone is on the same page. It fosters collaboration between teams, allowing designers to communicate their ideas clearly, developers to ask questions about technical feasibility, and stakeholders to provide input on the project’s progress. Having sketches as a reference point helps keep everyone aligned throughout the lifecycle of the project.On a practical level, converting sketches into code involves selecting the best programming languages, frameworks, and tools that suit the project’s needs. Developers take the visual elements and translate them into code, ensuring that the application works as expected and performs well. This involves not just writing the code for visual elements but also integrating them with other parts of the system, like databases or APIs. The process requires developers to pay attention to both the technical details and the user experience. They must ensure that the final product not only meets the technical specifications but also provides an interface that is easy to use and navigate.
    
    Another important aspect of the sketch-to-code process is the flexibility it offers in development. By working from the sketches, developers can follow an iterative development approach. This means that the project can evolve over time, based on feedback from users or stakeholders. As feedback is gathered, developers can make improvements to the software, refining it until it meets the needs of the users and aligns with the vision of the project. This iterative approach allows for continuous improvements, making the development process more adaptive and responsive to changing requirements or new insights.
    Ultimately, the process of translating sketches into code plays a crucial role in ensuring the success of a computer science project. It helps developers create software that is functional, meets the requirements, and provides a positive user experience. By starting with a clear, visual design, the project is guided in a consistent and structured way, reducing the chances of miscommunication or errors. Developers are able to work more efficiently, as they have a solid foundation to build on. And because sketches facilitate collaboration and communication, all stakeholders can be involved in the development process, ensuring that the final product is exactly what the user needs.
    
    
    \pagebreak
\section{\MakeUppercase{Literature Review}}    
\subsection{Different methods used for this kind of task}
    \sloppy  
The literature review for this research was conducted by searching 
through various academic repositories such as IEEE Xplore, Google 
Scholar, and other trusted sources. The search included a variety 
of keywords such as "sketch to code," "image to code," "GUI to 
functional code," and others, all relating to the process of 
converting visual design representations into executable software
 code. The review primarily focused on research papers, conference
  proceedings, and relevant articles that detail the methodologies,
   tools, and technologies used in this field. Along with the academic
    papers, various existing systems that are attempting to automate 
    this process were also reviewed to understand the evolution and
     challenges faced in the area of sketch-to-code conversion.

One of the most notable contributions in this area comes from the
 2017 paper \cite{beltramelli2017pix2codegeneratingcodegraphical} which
  introduced Pix2code. Pix2code is a pioneering deep learning-based
 system designed to convert graphical user interfaces (GUIs) into
 executable code. By analyzing screenshots of a GUI, Pix2code identifies
 various components such as buttons, menus, and images, and generates
 corresponding code in different programming languages. This system 
 bridges the gap between design and development, making it easier 
 for designers to translate their visual ideas into working software 
 components. The key strength of Pix2code lies in its ability to automate 
 code generation, saving valuable time during the prototyping phase of development.
 However, the system faces challenges such as the need for large, high-quality datasets
  for training and issues in handling highly complex or non-standard designs. Despite
these hurdles, Pix2code remains a significant step forward, making the design-to-
development process faster and more efficient by reducing manual coding efforts.
 The use of deep learning models in Pix2code has been recognized for transforming
  how the design-to-development workflow is approached. 

Moving forward, in 2019, the paper \cite{10.1145/3290607.3312994} introduced a new approach to transforming hand-drawn sketches into 
interactive software prototypes. Unlike traditional methods of prototyping, 
which rely on manual coding or graphic design tools, Eve aimed to automate much 
of the process by utilizing \gls{ml} and image recognition algorithms. Eve’s 
approach involves several stages, including User Interface Sketch Recognition, 
Prototype Generation, and User Interaction, all of which contribute to generating 
a functional prototype from a simple sketch. By leveraging ML algorithms, the system 
recognizes UI components in hand-drawn sketches and transforms them into working 
software prototypes. This system provides a faster alternative to traditional prototyping
 methods, significantly reducing development time. However, while Eve presents an 
 innovative solution, the paper highlights challenges such as achieving recognition 
accuracy, handling the complexity of real-world UI designs, and ensuring the system's
 scalability and usability in real-world applications. Despite these challenges, the 
 paper presents a promising methodology for automating the sketch-to-prototype process, 
 which can greatly benefit the software development community by enabling faster 
 iterations and reducing dependency on expert coding skills. 

In 2020, another groundbreaking paper \cite{9971204}took a step further in automating the sketch-to-code process by focusing on low-fidelity, hand-drawn sketches of GUIs. The system utilizes advanced deep learning models, particularly convolutional neural networks (CNNs) for image recognition, and Long Short-Term Memory (LSTM) networks for sequence prediction. The goal of this research was to improve the accuracy and efficiency of converting simple sketches into functional code. CNNs are used to recognize and identify different UI elements, while LSTMs help generate the corresponding code sequences needed for the final output. By combining these two techniques, the system is better equipped to handle complex UI layouts and intricate designs, which earlier models struggled to interpret correctly. This model also aims to handle dense and detailed sketches, where the layout and UI components may overlap or be difficult to distinguish. However, while the approach provides significant improvements, the system still faces issues in interpreting diverse sketch styles and integrating the generated code into existing development workflows. These challenges point to the complexity of designing a system that can accurately process a wide range of sketch styles while producing functional, high-quality code. Despite the challenges, the research represents an important step in improving the process of generating code from graphical user interfaces and pushing the boundaries of sketch-to-code systems. 

Building on these advancements in 2021, the paper \cite{Baule2021} introduced a system focused on mobile app development. The aim was to allow users to sketch mobile app interfaces, which would then be automatically converted into functional code. The research aimed to democratize the development process, enabling users without programming knowledge to create mobile applications. This system uses CNNs for accurate recognition of various UI elements and LSTMs to generate the code that corresponds to the design. The end goal of this research is to empower end-users by providing them with tools to create applications without needing to learn complex coding languages. However, similar to earlier models, this system faces challenges, such as accurately recognizing different sketch styles, ensuring the generated code functions as expected, and integrating it seamlessly into mobile app development environments. While the research marks an important step in enabling non-technical users to participate in mobile app creation, the system's challenges highlight the complexities of converting hand-drawn sketches into functional code that can be used in real-world development. Nonetheless, it represents a significant breakthrough in user-friendly development tools for mobile apps. 

Similarly in 2022, the paper \cite{2211.14607} made a significant contribution to the field of automated code generation. This research proposed a multi-step framework designed to generate skeleton code for full-stack applications from hand-drawn sketches. The framework incorporated advanced deep learning techniques, including convolutional neural networks (\gls{cnn}), to identify and classify user interface (\gls{ui}) components such as buttons, text fields, and menus. It also leveraged advanced image processing algorithms to enhance the accuracy of element recognition. Once the UI elements were identified, the system mapped these components to corresponding front-end and back-end code modules.The innovation of Sketch2FullStack lay in its ability to streamline the development process by bridging the gap between conceptual design and implementation. The generated skeleton code provided developers with a strong foundation, reducing the time spent on initial setup and allowing for greater focus on customization and functionality. The research addressed challenges such as handling complex layouts and integrating multiple layers of code across the stack, from front-end HTML and CSS to back-end APIs and databases. However, the system encountered limitations in interpreting sketches with intricate designs or unconventional layouts, highlighting the need for further advancements in deep learning algorithms. This work demonstrated how automated tools could significantly improve developer productivity while maintaining alignment with design specifications.

Moreover in 2023, the paper \cite{2302.06144} introduced a novel methodology for transforming sketches into executable code. Unlike prior systems, SkCoder emulated the behavior of developers by utilizing an intelligent code-retrieval approach. The system was designed to search for similar code snippets in large-scale repositories, extract relevant parts, and adapt them to match the structure and style of the sketch provided by the user. SkCoder combined machine learning models with heuristic-based rules to refine the retrieved code and ensure its functionality within the project's context.A key feature of SkCoder was its ability to incorporate user feedback iteratively, allowing developers to fine-tune the generated code. This feedback loop enabled the system to learn from corrections, progressively improving its accuracy and relevance. The research highlighted the importance of balancing automated generation with developer intervention to address edge cases and maintain code quality.Additionally, SkCoder placed a strong emphasis on enhancing collaboration between designers and developers. By providing a semi-automated solution, the tool enabled designers to visualize how their sketches would translate into actual code while offering developers a solid starting point for implementation. The system demonstrated particular effectiveness in handling standard UI components and layouts but faced challenges when dealing with non-standardized or overly creative designs. Nonetheless, SkCoder represented a significant step forward in creating intelligent and adaptable tools for modern software development.

More recently, in 2024, a paper \cite{10537336} was published, offering another perspective on the sketch-to-code problem by focusing on the generation of HTML and CSS code from hand-drawn sketches. This paper addresses the need for automation in web development by proposing a system that uses deep learning techniques to convert hand-drawn UI sketches into working web pages. The system goes through a four-phase process: pre-processing, segmentation, feature extraction, and classification, with each phase designed to interpret the visual features of the sketch and categorize them into HTML and CSS elements. \gls{ldps} are used in the feature extraction phase to capture detailed visual patterns, enhancing the system's ability to correctly classify UI components. While earlier methods in this domain were based on rule-based systems, which often lacked flexibility, this new system introduces a more robust and adaptable approach. However, like its predecessors, it faces challenges related to scalability, accuracy, and handling diverse sketch styles. Despite these challenges, the paper provides an important contribution to the ongoing development of systems capable of automatically converting sketches into functional web code. 

\subsection{Vision Transformer for Computer vision}
In addition to the above works, a significant contribution to the field of computer vision was made by Dosovitskiy et al. in 2021 with their paper \cite{dosovitskiy2021imageworth16x16words} This research introduced the \gls{vit}, a novel deep learning model designed to process images in a way similar to how \gls{nlp} models handle text. Unlike traditional \gls{cnn}, which use filters to process images, the ViT model treats images as a sequence of patches, each of which is classified using transformer models. The ViT model has achieved outstanding results in large-scale image classification tasks, and its principles have been adapted for use in sketch-to-code systems. By using transformers, researchers have been able to improve the accuracy of image recognition tasks and enhance the ability of sketch-to-code systems to interpret complex visual data. While the Vision Transformer model holds great potential for improving sketch-to-code systems, challenges still remain in fine-tuning the model to process UI designs accurately and ensuring it integrates seamlessly with the code generation process. 

In conclusion, the literature reveals substantial progress in the development of systems that 
automate the translation of sketches into executable code. From the early rule-based methods 
to more recent deep learning-based approaches, significant strides have been made in enhancing
 the accuracy, scalability, and efficiency of these systems. Despite the advancements,
challenges persist, particularly in handling diverse sketch styles, improving recognition
accuracy, and ensuring seamless integration with real-world development workflows. 
Nonetheless, the research in this domain continues to evolve, with new methodologies and models.
\pagebreak
\section{\MakeUppercase{Requirement Analysis}}
    \subsection{Software Requirement}
    In order to implement our project of automating the process of converting sketches into functional front-end code, the following software tools and libraries are indispensable. Each of these tools has been chosen for its critical role in different stages of the project:
    \subsubsection{Python}
Python is a high-level, versatile programming language that has become a cornerstone in fields such as artificial intelligence, machine learning, data science, and software development. Its design philosophy emphasizes readability and simplicity, enabling developers to focus on solving problems rather than managing code complexity. Python boasts an extensive ecosystem of libraries and frameworks that make it highly adaptable for implementing cutting-edge solutions.In this project, Python serves as the foundation for every aspect of development, from preprocessing sketch data to training machine learning models and generating functional code. Its cross-platform compatibility allows smooth integration with various tools and frameworks. Furthermore, Python’s active open-source community ensures regular updates, abundant resources, and support, making it ideal for building a robust and scalable solution.
\subsubsection{Numpy}
\gls{numpy} is a core library for numerical and scientific computing in Python. It is designed to provide efficient support for handling large, multi-dimensional arrays and matrices, which are essential for processing image data and performing computations in machine learning workflows. NumPy's optimized performance ensures that computational tasks involving large datasets can be executed quickly and reliably.In our project, NumPy plays a significant role in processing images of hand-drawn sketches. It enables efficient manipulation of pixel data, normalization of input images, and transformations such as resizing and cropping. Additionally, NumPy is a critical component in data preparation, where it facilitates mathematical operations that prepare data for feeding into deep learning models. Its seamless integration with libraries like TensorFlow further simplifies the workflow and ensures consistency.
\subsubsection{Tensorflow}
TensorFlow is an open-source machine learning framework developed by Google. It is one of the most widely used platforms for building, training, and deploying machine learning and deep learning models. TensorFlow provides powerful tools for managing tensors, which are multi-dimensional arrays that serve as the backbone of neural network operations. It supports automatic differentiation, optimization, and distributed training, making it a versatile choice for handling large-scale projects.For this project, TensorFlow acts as the core engine for implementing the deep learning models that drive the conversion of sketches into functional code. It allows us to design sophisticated convolutional neural networks (\gls{cnn}) to identify UI elements in sketches, and sequence models like \gls{rnn} or transformers for generating structured code. TensorFlow's scalability enables us to train models on high-performance hardware such as GPUs, ensuring the project can handle the computational demands of large datasets and complex designs. Additionally, TensorFlow's deployment tools make it easier to integrate the trained models into production environments.
\subsubsection{Keras}
Keras is a high-level neural network \gls{api} built on top of TensorFlow, offering a simplified and user-friendly interface for creating and training deep learning models. Keras provides pre-built components such as layers, optimizers, and loss functions, making it accessible even for developers who are not experts in machine learning. Its modular structure allows users to experiment with different architectures and hyperparameters with minimal effort.In the context of this project, Keras is essential for building and fine-tuning neural networks designed to interpret sketch data and generate front-end code. It allows for rapid prototyping and testing of models, enabling developers to iterate quickly and find optimal solutions. Keras also includes tools for visualizing model performance, such as training accuracy and loss, which are crucial for diagnosing issues and improving model reliability. By leveraging Keras, we can streamline the development of deep learning pipelines and maintain focus on achieving project goals without being bogged down by implementation details.
\pagebreak
\subsection{Hardware Requirements}  
The hardware requirements for this project are essential to ensure the effective collection of input data and the efficient training of deep learning models. Two primary hardware components are required: a camera device for capturing sketched designs and a cloud computing platform for performing computationally intensive training tasks. Together, these components form the backbone of the system's functionality and performance.  

\subsubsection{Camera Device}  
For the successful digitization of manually sketched designs, a high-quality camera device is required. The minimum recommended resolution for the camera is 12 \gls{mp}, which is commonly found in most modern smartphones. Such devices are capable of producing clear and detailed images that are sufficient for analysis and feature extraction. However, using a camera with a higher resolution, such as 48 \gls{mp} or 108 \gls{mp}, can further enhance the quality of the images, capturing fine details of intricate sketches that may otherwise be lost. In professional settings, \gls{dslr} or mirrorless cameras can be employed to achieve even better results. These advanced cameras offer superior image quality and features like manual focus, adjustable exposure, and higher dynamic range. Additionally, features such as image stabilization and customizable settings for white balance and ISO ensure distortion-free and sharp images, even under challenging lighting conditions. To maximize accuracy, it is recommended to capture images in a well-lit environment, as proper lighting significantly improves the visibility and clarity of sketch details.  

\subsubsection{Cloud Computing Platform}  
The training of deep learning models for this project requires computational resources beyond what standard personal computers can provide. A robust cloud computing platform with access to dedicated \gls{gpu} and \gls{cpu} resources is essential for handling large datasets and performing complex operations. The recommended platforms include Google Colab and Kaggle, which provide free hosted environments equipped with powerful \gls{gpu} resources, such as NVIDIA Tesla T4 or K80. These platforms are well-suited for deep learning tasks and allow developers to focus on model building and experimentation without worrying about hardware constraints. Google Colab offers seamless integration with Google Drive, allowing easy storage and retrieval of datasets and model checkpoints. Its support for collaboration makes it an ideal choice for teams working on the project. Similarly, Kaggle provides access to community-shared datasets and pre-trained models, which can accelerate development and reduce the time required for data preparation.  

For larger-scale projects or cases where free resources are insufficient, paid platforms like AWS, Google Cloud Platform, or Microsoft Azure can be utilized. These platforms offer flexible configurations, including high-performance \gls{gpu} instances like NVIDIA V100 or A100, enabling faster and more efficient model training. Additionally, they provide advanced features such as automated scaling and integration with machine learning pipelines, ensuring that the system can adapt to evolving requirements during development. By leveraging these hardware components effectively, the project can achieve high accuracy in sketch analysis and efficient training, ensuring the final system meets both technical standards and user expectations. 
    \pagebreak
 
